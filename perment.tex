%%%%%%%%%%%%%%%%%%%%%%%%%%%%%%%%%%%%%%%%%
% Simple Article
% Integrated article template with simple for make4ht
% LaTeX Class
% Version 1.0 (10/11/20)
%
% This class originates by:
% Vel and  Nicolas Diaz
%
% Authors:
% Muhammad Uliah Shafar
%
%
% Free License:
%
%
%%%%%%%%%%%%%%%%%%%%%%%%%%%%%%%%%%%%%%%%%
\documentclass[11pt]{simart} % Font size (can be 10pt, 11pt or 12pt)

%----------------------------------------------------------------------------------------
%	TITLE SECTION
%----------------------------------------------------------------------------------------
% MAIN TITLE SECTION
\title{
\textbf{Personal Statement} \\
} % Title and subtitle
%\date{\textbf{\DTMtoday}}
\date{\textbf{\today}}
\author{Uliah}

%----------------------------------------------------------------------------------------
% OTHER TITLE SECTION

%\title{\textbf{Sistem Sarana dan Prasarana Jl. Pinggir Laut} \\ {\Large\itshape Infrastructure of Waterfront Parepare City}} % Title and subtitle

%\author{\textbf{Uliah Shafar} \\ \textit{Universitas Diponegoro}} % Author and institution

%\date{\today} % Date, use \date{} for no date

%----------------------------------------------------------------------------------------



\begin{document}
\maketitle % Print the title section

%----------------------------------------------------------------------------------------
%	ESSAY BODY
%----------------------------------------------------------------------------------------
\section{Motivations with which you apply for this program}
\section*{What has motivated you to pursue a graduate degree? Why are you interested in your major?}

Growing in small city was a wonderful gift. It was free of congestion, environmental pollution and overpopulation. Parepare is a small city where I was raised by my parents with three little sibling. Even though a small city, this city was where the third president of Indonesia, BJ Habibie, was born.
Moreover, it owns uncomparable attraction itself because it is bordering straight to the coast and highland.

Beside its innate attraction, the city today has new facilitation that potentially become a primary support of urban residents' activities.
Up to this point, Parepare has been connected to the train transportation moda through the capital city of South Sulawesi.
Up to this point, a new train transportation moda that has connected to the capital city of south sulawesi has just launched in the end of last year.

Additionally, it has just rejuvenated its soccer stadium that has been used by soccer club for the primary Indonesian's soccer league.

This advancement h


Improving the quality of architecture education in my hometown is best respond towards the city's potential of becoming urban centre.

I was growing in up small city named Parepare.

crucial
push
An impact of an increasing development in a small city is like a double-edge sword.
%However, the knowledge of architecture and city planning are useful for making prevention act from its downward impact.
However, the knowledge of an architecture and city planning in some degree can prevented a city from the bad side of the issue.

Stakeholders or governors who have a certain amount of knowledge, undoubtly, will have a higher chance to suppress the bad side of the raising development.

its negative side can be prevented with certain knowledge such as architecture or city planning.

Accusation of

This prevention can be achieve by some people who might be stakeholders or governors with an adequate experience and knowledge. But for me,

I was raised in small city of Parepare by my parents with three little sibling. The city was a nice place to grow because it was free of congestion, environmental pollution and overpopulation. Regardless it is a small town, the city was a place

Growing in small city was a wonderful gift. It was free of congestion, environmental pollution and overpopulation. That is true, Parepare is a small city where I was raised by my parents with three little sibling. Although it is a small city, this city is the birth place of the third president of Indonesia, BJ Habibie. And also it has attractiveness itself because it borders with sea and hinglands directly.

-- bring education forward
-- bring research of indonesia forward.

of Parepare by my parents with three of my siblings.
The city was free

\section{Educational background}
\section*{cerita masa kuliah}

\section{Significant experiences you have had; persons or events that have had a significant influence on you}
\section*{How will your background and preparation, including education and professional experience, contribute to your success in the graduate program? Please describe any challenges you have faced during your previous education and how to overcome your challenges.}

\section{Extracurricular activities such as club activities, community service activities or work experiences}
\section*{pengalaman terkait / prestasi terkait}
When I was in high school, I was active on school organizations, especially PMR (youth red cross). At the time, we managed some organization events. The most impressive events were first, the time we were organizing blood donor in the school that was attended by few of students and teachers. Second, when we were arranging trainings for potential members for example, the first aid, the use of general medical equipment, and SAR (search and rescue) training.

In a college, I was involved in architecture student organization outside campus named PAMIY. At the moment, we arranged few events which involved students from different Yogyakarta university. First, we were designing and constructing a small library for village's small mosque in the Parangritis village, Yogyakarta. Second, we developed a number of sport and art competitions among architecture students of Yogyakarta. Beside organization's activity, I have been an assistant lecture for the course of DED (detail engineering drawing). I aid students to apply the theory that was just explained. Sometimes, there were  students who need furhter counseling, therefore I usually took the time to teach them extensively in order to catch up with other peers.

When I was in a postgraduate, I was particularly active on writing's activity than involve in organization. The reason was that the occurrence of virus which had hindered most of usual activities. Nevertheless, it allowed me to earn 4.00 point of GPA and graduated with high honors. Besides, I could also publish an article in reputated national journal. All my writings including my thesis is all about public space which I always concern about because it relates to the human's live including social, health, and even wellbeing.

In semester 3 of my master's degree, I had a chance to work in school renovation projects while I was developing my proposal thesis. This renovation demanded for an addition of a inclusive design and a garden. Thus, as one of drafters, I was supposed to create designs for a ramp and a garden in front of classes as requested in location survey early in the beginning of project. %give some conclusion

By the time I graduated from Diponegoro University, a few months later I was appointed as permanent lecture in Muhammadiyah University of Parepare.
This university had just openned a department of the city and regional planning which needed potential members of lectures to run it.
I have been trusted to deliver lectures in three subjects in this year terms, although it has not been too active because of little enthusiasts of the course.
Therefore, in meantime, I also put an effort in promoting this department to propective students with social media and any other platform.
In additon to lecturing, I have attended two days scientific writing workshop that would increase the writing skill of fresh lecturers. In this workshop, I had offered a proposal with my lecturer coworker for hoping my proposed research would be funded.


\section{If applicable, describe awards you have received, publications you have made, or skills you have acquired, etc.}

\section{What particular research interests do you hope to explore? Why are you interested in these areas? What do you hope to achieve in your graduate program?}
\section*{ketertarikan pada jurusan}
\section*{alasan memilih jurusan/univ/korea}
\section*{pengalaman berhubungan dengan kroea}















%----------------------------------------------------------------------------------------
%	BIBLIOGRAPHY
%----------------------------------------------------------------------------------------

%\bibliographystyle{apalike}

%\bibliography{biblio.bib}



\end{document}
