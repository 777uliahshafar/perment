%%%%%%%%%%%%%%%%%%%%%%%%%%%%%%%%%%%%%%%%%
% Simple Article
% Integrated article template with simple for make4ht
% LaTeX Class
% Version 1.0 (10/11/20)
%
% This class originates by:
% Vel and  Nicolas Diaz
%
% Authors:
% Muhammad Uliah Shafar
%
%
% Free License:
%
%
%%%%%%%%%%%%%%%%%%%%%%%%%%%%%%%%%%%%%%%%%
\documentclass[11pt]{simart} % Font size (can be 10pt, 11pt or 12pt)

%----------------------------------------------------------------------------------------
%	TITLE SECTION
%----------------------------------------------------------------------------------------
% MAIN TITLE SECTION
\title{
\textbf{Personal Statement} \\
} % Title and subtitle
%\date{\textbf{\DTMtoday}}
\date{\textbf{\today}}
\author{Uliah}

%----------------------------------------------------------------------------------------
% OTHER TITLE SECTION

%\title{\textbf{Sistem Sarana dan Prasarana Jl. Pinggir Laut} \\ {\Large\itshape Infrastructure of Waterfront Parepare City}} % Title and subtitle

%\author{\textbf{Uliah Shafar} \\ \textit{Universitas Diponegoro}} % Author and institution

%\date{\today} % Date, use \date{} for no date

%----------------------------------------------------------------------------------------



\begin{document}
%\maketitle % Print the title section

%----------------------------------------------------------------------------------------
%	ESSAY BODY
%----------------------------------------------------------------------------------------
%\section{Motivations with which you apply for this program}
%\section*{What has motivated you to pursue a graduate degree? Why are you interested in your major?}

Growing in a small city was a wonderful gift. It was free of a congestion, an environmental pollution and an overpopulation. Parepare is a small city where I was raised by my parents with three little siblings. Even though a small city, this city was where the third president of Indonesia, BJ Habibie, was born.
Moreover, it owns incomparable attractions itself because it is bordering straight to the coast and highland.

Beside its natural attractions, the city today has new facilitation that potentially become a primary support of the urban residents' activities.
Up to this point, a new train transportation that connects to the capital city of the South Sulawesi Province has just launched recently.
And also, it has just renovated its soccer stadium that has been used by the local soccer club for the primary Indonesian's soccer league.
This development, certainly, is not only beneficial to the local residents but also attracts the outsiders which leaves the urbanization issues to the city.

According to Central Bureau of Statistic, Parepare has shown an increase of population from the year 2019 amounts to 5\% in 2021. This data has been supported by some developments that has occurred over the course of years particularly in outskirts of Parepare city. For example, the housing's developments are widely constructed in certain spots of city's outskirts rapidly. And also a dozen of public spaces or building constructions have been took place in the city's center.

Improving the quality of the architectural education in my hometown is the best respond towards the urbanization issue.
This issue is like a double-edge sword. In the one hand, it can drives the city to a successful development in many sectors such as an economic growth. In the other hand, it can also probably brings deteriorated impacts such as slum residential, squatter developments and terrible planning of the city. The knowledge from an architectural education hopefully will be used by some policy makers, developers, and even architects to address those impacts practically and accurately.

University is a right place to start improving the architectural education in a city.
But, Parepare only houses to one architectural university department in entire city.
This department is in Muhammadiyah University of Parepare under the engineering faculty, namely the City and Regional Planning field of study.
It was just founded a couple years ago and has been running in slow pace. This, certainly would hinder the improvement of education.

Lecturers essentially are the pioneer to increase the quality of university's education.
If the lectures were given a proper and best education, they would be able to deliver lecture more comprehensively.
Therefore, it will produces graduates who have such skill to handle any real problems out there including urbanization through their knowledge and experience.

My intention to continue a doctorate degree will allow me to be apart of a respond of issue.
Doctorate degree, undoubtedly has capability to enhance the sensitivity towards humanity and environmental issues.
In addition, it would offers such experiences, knowledge and networking that are well-established.
This would prepared me to improve the education based on the broad network of architecture and researcher. The examples are probably delivering lectures or writing publications.


%\noindent\rule{\linewidth}{0.5pt}
%\section{What particular research interests do you hope to explore? Why are you interested in these areas? What do you hope to achieve in your graduate program?}
%\section*{ketertarikan pada jurusan}
%\section*{alasan memilih jurusan/univ/korea}
%\section*{pengalaman berhubungan dengan kroea}
In my pursuit of a doctorate, I would like to address urbanization issues by writing a dissertation about  a public space identity. In fact, Urbanization  could cause a fast development which would have an impact on the change and the loss of local identity and the emergence of standardized and homogeneous public spaces' character.  This dissertation is hoped to understanding the identity of all public spaces through identify characters of each public spaces. The understanding of places' identities would be explored through the lens of Korean architecture cases. The result of this research would heal one aspect of the urbanization issues by providing better places to recover and develop people lives in the city.

Before developing my dissertation, absolutely I would take some courses. These courses include an architecture design, an urban architecture and regeneration, a sustainable architecture and a urban design courses.
The Architecture design, one of the course, indeed has covered an area on creating architecture design reflecting local characters which in line with my dissertation problem. Therefore, these planned course would ready myself to writing the dissertation.

I believe that I could pursue my doctorate with the support of the Global Korea Scholarship (GKS) in Korea. The reason is that universities in Korea have an advanced and comprehensive courses for a architecture program. Another reason is that Korea as a country successfully develop its city and architecture.
For example, first, the success transformation of Cheonggyecheon stream as a attraction of public in the city center which I learn from an undergraduate seminar.
Second, the easy access of transportation in the city of Seoul which I knew from one of my college there. These reason and many more have indicated Korea is best place to learn architecture.

%\noindent\rule{\linewidth}{0.5pt}
%\section{Educational background}
%\section*{cerita masa kuliah}

Not only GKS could support me, but I hope my education would enable me to go a little step forward. I began to study architecture in the vocational high school named SMKN 2 Parepare. I learned some skills there including a manual drawing, CAD, and 3D modeling. Then, I continue to University Technology of Yogyakarta to pursuit my undergraduate degree. After 5 years studying in-depth knowledge of the architectural theory and skill, I entered the Diponegoro University to continue my study in a master's degree. In developing my thesis, I acquired a latex skill in order to ease my writing endeavor beside theoretical subjects that I had learned.
This decade of studying architecture has somewhat empower me to keep developing myself.


%\noindent\rule{\linewidth}{0.5pt}
%\section{Significant experiences you have had; persons or events that have had a significant influence on you}
%\section*{How will your background and preparation, including education and professional experience, contribute to your success in the graduate program? Please describe any challenges you have faced during your previous education and how to overcome your challenges.}

This study of course is not come without any struggle and not so good experiences. For instance, I used to late on collecting my proposal of final project. It had made me to repeat one phase of a final project's proposal (half semester). This experience is one of my weakness's consequence which perfectionism. I always have perfected my assignment to the core until I denied the deadline. Therefore, in the future I will find the right balance between perfection and time.

%\noindent\rule{\linewidth}{0.5pt}
%\section{Extracurricular activities such as club activities, community service activities or work experiences}
%\section*{pengalaman terkait / prestasi terkait}
%\section{If applicable, describe awards you have received, publications you have made, or skills you have acquired, etc.}
We are all agree that education is not only taught in formal institutions. Organization, volunteering and even workplace could provide much broader education. For example, I learned a team work and a compassion in achieving teams' goals. Personally, I have some of those experiences which I will tell chronologically as follows.

When I was in a high school, I was active on school organizations, especially the PMR (youth red cross). At the time, we managed some organization events. The most impressive events were first, the time we were organizing a blood donor event in the school that was attended by few of students and teachers. Second, when we were arranging training for potential members for example, the first aid technique, the use of general medical equipment, and a SAR (search and rescue) training.

In a college, I was involved in a architecture student organization outside campus named PAMIY. At the moment, we arranged few events which involved students from different Yogyakarta University. First, we were designing and constructing a small library for the village's small mosque in the Parangritis village, Yogyakarta. Second, we developed a number of sport and art competitions among architecture students of Yogyakarta. Beside organization's activities, I had been an assistant lecture for a course of DED (detail engineering drawing). I aided students to apply the theory that was just explained. Sometimes, there were  students who need further counseling, therefore I usually took the time to teach them extensively in order to catch up with other peers.

When I was in a postgraduate, I was particularly active on writing's activities than involve in organization. The reason was that the occurrence of virus which had hindered most of usual activities. Nevertheless, it allowed me to earn 4.00 point of GPA and graduated with high honors. Besides, I had also published an article in a reputable national journal. All my writings including my thesis are all about public space which I always concern about because it relates to the human's live including social, health, and even well being.

In semester 3 of my master's degree, I had a chance to work in school renovation projects while I was developing my proposal thesis. This renovation demanded for an addition of a inclusive design and a garden. Thus, as one of drafters, I was supposed to create designs for a ramp and a garden in front of classes as requested in the location survey early in the beginning of project. %give some conclusion

By the time I graduated from the Diponegoro University, a few months later I was appointed as a permanent lecture in the Muhammadiyah University of Parepare.
This university had just opened a department of the city and regional planning which needed potential members of lectures to run it.
I have been trusted to deliver lectures in three subjects in this year terms, although it has not been too active because of little potential student of the course.
Therefore, in meantime, I also put an effort in promoting this department to prospective students with social media and any other platforms.
In addition to lecturing, I have attended two days scientific writing workshop that would increase the writing skill of fresh lecturers like me. In this workshop, I had offered a proposal with my lecturer coworker for hoping my proposed research would be funded.

These experiences, certainly, will not be a weapon only for completing my higher degree, but also it would prepared me to be a good potential GKS awardee. I would be easily able to adapt to a new environment, survive in any circumstance and fulfill GKS expectations. To conclude, although I haven't see Korea in real life , I believe the chance of study in Korea is a life changer for me.















%----------------------------------------------------------------------------------------
%	BIBLIOGRAPHY
%----------------------------------------------------------------------------------------

%\bibliographystyle{apalike}

%\bibliography{biblio.bib}



\end{document}
