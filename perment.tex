%%%%%%%%%%%%%%%%%%%%%%%%%%%%%%%%%%%%%%%%%
% Simple Article
% Integrated article template with simple for make4ht
% LaTeX Class
% Version 1.0 (10/11/20)
%
% This class originates by:
% Vel and  Nicolas Diaz
%
% Authors:
% Muhammad Uliah Shafar
%
%
% Free License:
%
%
%%%%%%%%%%%%%%%%%%%%%%%%%%%%%%%%%%%%%%%%%
\documentclass[11pt]{simart} % Font size (can be 10pt, 11pt or 12pt)

%----------------------------------------------------------------------------------------
%	TITLE SECTION
%----------------------------------------------------------------------------------------
% MAIN TITLE SECTION
\title{
\textbf{Personal Statement} \\
} % Title and subtitle
%\date{\textbf{\DTMtoday}}
\date{\textbf{\today}}
\author{Uliah}

%----------------------------------------------------------------------------------------
% OTHER TITLE SECTION

%\title{\textbf{Sistem Sarana dan Prasarana Jl. Pinggir Laut} \\ {\Large\itshape Infrastructure of Waterfront Parepare City}} % Title and subtitle

%\author{\textbf{Uliah Shafar} \\ \textit{Universitas Diponegoro}} % Author and institution

%\date{\today} % Date, use \date{} for no date

%----------------------------------------------------------------------------------------



\begin{document}
\maketitle % Print the title section

%----------------------------------------------------------------------------------------
%	ESSAY BODY
%----------------------------------------------------------------------------------------
\section{Motivations with which you apply for this program}
\section*{What has motivated you to pursue a graduate degree? Why are you interested in your major?}

Growing in small city was a wonderful gift. It was free of congestion, environmental pollution and overpopulation. Parepare is a small city where I was raised by my parents with three little sibling. Even though a small city, this city was where the third president of Indonesia, BJ Habibie, was born.
Moreover, it owns incomparable attraction itself because it is bordering straight to the coast and highland.

Beside its natural attraction, the city today has new facilitation that potentially become a primary support of urban residents' activities.
Up to this point, a new train transportation mode that connects to the capital city of south Sulawesi has just launched recently.
And also, it has just rejuvenated its soccer stadium that has been used by local soccer club for the primary Indonesian's soccer league.
This development, certainly, is not only beneficial to the local residents but also attracts outsider which leave urbanization issues to the city.

According to Central Bureau of Statistic, Parepare has shown an increase of population from the year 2019 amounts to 5\% in 2021. This data has been supported by some developments that has occurred over course of years particularly in outskirts of Parepare. For example, the housing's developments are widely constructed in certain spot of city's outskirts rapidly. And also dozen of public space or building constructions have been took place in the city's center.

Improving the quality of architectural education in my hometown is the best respond towards the urbanization issue.
This issue is like a double-edge sword. In the one hand, it can drives the city to a successful development in many sectors such as an economic growth. In the other hand, it can also probably bring deteriorated impacts such as squatter developments and terrible planning of the city. The knowledge from architectural education hopefully will be used by some policy makers, developers, and even architects to address those impacts practically and accurately.

University is a right place to start improving the architectural education in a city.
But, Parepare only house to one architectural university department in entire city.
This department is in Muhammadiyah University of Parepare under engineering faculty, namely City and Regional Planning field of study.
It was just founded a couple years ago and has been running in slow pace. This, certainly would hinder the improvement.

Lecturers essentially are the pioneer to increase the quality of university's education.
If the lectures were given proper and best education, they would be able to deliver lecture more comprehensively.
Therefore, it will produce graduates who have such skill handle any real problems out there including urbanization through their knowledge and experience.

My intention to continue doctorate degree will allow me to be apart of respond of issue.
Doctorate degree, undoubtedly has capability to enhance the sensitivity towards humanity and environmental issues.
In addition, it would offers such experience, knowledge and relation that are well-established.
This would prepared me to improve the education based on broad network of architecture and researcher. The examples are probably delivering lecture or writing publication.

In my pursuit of doctorate, I would like to address urbanization issue by writing dissertation about public space identity. In fact, Urbanization  could cause fast development which would have impact on the change and the loss of local identity and the emergence of standardized and homogeneous city's character.  This dissertation is hoped to understanding identity of all public spaces through identify character of each public spaces. The understanding of places' identity would be explored through the lens of Korean architecture cases. The result of this research would heal one aspect of the urbanization issue by providing better places to recover and develop people lives in the city.

Before developing my dissertation, absolutely I would take some courses. These courses include architecture design, urban architecture and regeneration, sustainable architecture and urban design courses.
Architecture design, One of the course, indeed has covered an area on creating architecture design reflecting local character which in line with my dissertation problem. Therefore, these planned course would ready myself to writing the dissertation.

I believe that I could pursue my doctorate with the support of Global Korea Scholarship (GKS) in Korea. The reason is that university in korea have an advanced and comprehensive courses for architecture program. Another reason is that Korea as a country sucessfully develop its city and architecture.
For example, first, the success  transforamation of chunnam rver as public spaces attraction in city cnter which i learn from  undergraduate seminar.
Second, the easy access of transportation in the city of SEOUL which I knew from one of my college there. These reason and many more have indicated korea is best place to learn architecture.

Not only GKS could support me, but I hope my education would enable me to go little step forward.
First, I had learnt architectural theory and software such as a manual drawing, CAD, 3D modeling, and many subjects in vocational high school and undegraduate degree.
Second, I had gained more comprhensive knowledge and its practicallity in master degree with the accomplishment of my thesis. This experience indeed, had huge impact on my architectural view towrds the many issues that have developed in these days.








Korea is one of the best place to learn architecture as I am blown up by its architecture development. For example,


%They have all been taught in the few korea university such chonnam national university.

Joining Global Korea Schoalrship (GKS) program might be the best first step to actualize this belief.

The exposure of korea's experience and education would allow me to teach architecture extensively in the university I would work with.


My dream is to make architecture education  available in the Muhammadiyah Parepare Univeristy.


-- bring education forward
-- bring research of indonesia forward.

of Parepare by my parents with three of my siblings.
The city was free

\section{Educational background}
\section*{cerita masa kuliah}

\section{Significant experiences you have had; persons or events that have had a significant influence on you}
\section*{How will your background and preparation, including education and professional experience, contribute to your success in the graduate program? Please describe any challenges you have faced during your previous education and how to overcome your challenges.}

\section{Extracurricular activities such as club activities, community service activities or work experiences}
\section*{pengalaman terkait / prestasi terkait}
When I was in high school, I was active on school organizations, especially PMR (youth red cross). At the time, we managed some organization events. The most impressive events were first, the time we were organizing blood donor in the school that was attended by few of students and teachers. Second, when we were arranging trainings for potential members for example, the first aid, the use of general medical equipment, and SAR (search and rescue) training.

In a college, I was involved in architecture student organization outside campus named PAMIY. At the moment, we arranged few events which involved students from different Yogyakarta university. First, we were designing and constructing a small library for village's small mosque in the Parangritis village, Yogyakarta. Second, we developed a number of sport and art competitions among architecture students of Yogyakarta. Beside organization's activity, I have been an assistant lecture for the course of DED (detail engineering drawing). I aid students to apply the theory that was just explained. Sometimes, there were  students who need furhter counseling, therefore I usually took the time to teach them extensively in order to catch up with other peers.

When I was in a postgraduate, I was particularly active on writing's activity than involve in organization. The reason was that the occurrence of virus which had hindered most of usual activities. Nevertheless, it allowed me to earn 4.00 point of GPA and graduated with high honors. Besides, I could also publish an article in reputated national journal. All my writings including my thesis is all about public space which I always concern about because it relates to the human's live including social, health, and even wellbeing.

In semester 3 of my master's degree, I had a chance to work in school renovation projects while I was developing my proposal thesis. This renovation demanded for an addition of a inclusive design and a garden. Thus, as one of drafters, I was supposed to create designs for a ramp and a garden in front of classes as requested in location survey early in the beginning of project. %give some conclusion

By the time I graduated from Diponegoro University, a few months later I was appointed as permanent lecture in Muhammadiyah University of Parepare.
This university had just openned a department of the city and regional planning which needed potential members of lectures to run it.
I have been trusted to deliver lectures in three subjects in this year terms, although it has not been too active because of little enthusiasts of the course.
Therefore, in meantime, I also put an effort in promoting this department to propective students with social media and any other platform.
In additon to lecturing, I have attended two days scientific writing workshop that would increase the writing skill of fresh lecturers. In this workshop, I had offered a proposal with my lecturer coworker for hoping my proposed research would be funded.


\section{If applicable, describe awards you have received, publications you have made, or skills you have acquired, etc.}

\section{What particular research interests do you hope to explore? Why are you interested in these areas? What do you hope to achieve in your graduate program?}
\section*{ketertarikan pada jurusan}
\section*{alasan memilih jurusan/univ/korea}
\section*{pengalaman berhubungan dengan kroea}















%----------------------------------------------------------------------------------------
%	BIBLIOGRAPHY
%----------------------------------------------------------------------------------------

%\bibliographystyle{apalike}

%\bibliography{biblio.bib}



\end{document}
