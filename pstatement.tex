%%%%%%%%%%%%%%%%%%%%%%%%%%%%%%%%%%%%%%%%%
% Simple Article
% Integrated article template with simple for make4ht
% LaTeX Class
% Version 1.0 (10/11/20)
%
% This class originates by:
% Vel and  Nicolas Diaz
%
% Authors:
% Muhammad Uliah Shafar
%
%
% Free License:
%
%
%%%%%%%%%%%%%%%%%%%%%%%%%%%%%%%%%%%%%%%%%
\documentclass[11pt]{simart} % Font size (can be 10pt, 11pt or 12pt)

%----------------------------------------------------------------------------------------
%	TITLE SECTION
%----------------------------------------------------------------------------------------
% MAIN TITLE SECTION
\title{
\textbf{Personal Statement} \\
} % Title and subtitle
%\date{\textbf{\DTMtoday}}
\date{\textbf{\today}}
\author{Uliah}

%----------------------------------------------------------------------------------------
% OTHER TITLE SECTION

%\title{\textbf{Sistem Sarana dan Prasarana Jl. Pinggir Laut} \\ {\Large\itshape Infrastructure of Waterfront Parepare City}} % Title and subtitle

%\author{\textbf{Uliah Shafar} \\ \textit{Universitas Diponegoro}} % Author and institution

%\date{\today} % Date, use \date{} for no date

%----------------------------------------------------------------------------------------



\begin{document}
\maketitle % Print the title section

%----------------------------------------------------------------------------------------
%	ESSAY BODY
%----------------------------------------------------------------------------------------
 Motivations with which you apply for this program

An impact of an increasing development in a small city is like a double-edge sword.
%However, the knowledge of architecture and city planning are useful for making prevention act from its downward impact.
However, the knowledge of an architecture and city planning in some degree can prevented the bad side of the issue to the city.

Stakeholders or governors who have a certain amount of knowledge, undoubtly, will have a higher chance to suppress the bad side of the raising development.

its negative side can be prevented with certain knowledge such as architecture or city planning.

Accusation of

This prevention can be achieve by some people who might be stakeholders or governors with an adequate experience and knowledge. But for me,


I was raised in small city of Parepare by my parents with three little sibling. The city was a nice place to grow because it was free of congestion, environmental pollution and overpopulation. Regardless it is a small town, the city was a place


Growing in small city is a wonderful gift. It was free of congestion, environmental pollution and overpopulation. That is true, Parepare is a small city where I was raised by my parents with three little sibling. Although it is a small city, this city is the birth place of the third president of Indonesia, BJ Habibie. And also it has attractiveness itself because it borders with sea and hinglands directly.

-- bring education forward
-- bring research of indonesia forward.

of Parepare by my parents with three of my siblings.
The city was free

 Educational background
 Significant experiences you have had; persons or events that have had a significant influence on you
 Extracurricular activities such as club activities, community service activities or work experiences
 If applicable, describe awards you have received, publications you have made, or skills you have acquired, etc.

Note from univeristy

 What has motivated you to pursue a graduate degree? Why are you interested in your major?
 How will your background and preparation, including education and professional experience, contribute to your success in the graduate program? Please describe any challenges you have faced during your previous education and how to overcome your challenges.
 What particular research interests do you hope to explore? Why are you interested in these areas? What do you hope to achieve in your graduate program?

opening
cerita masa kuliah
ketertarikan pada jurusan
pengalaman terkati / presetasi terkait
alasan memilih jurusan/univ/korea
pengalaman berhubungan dengan kroea
closing/ kesimpulan













%----------------------------------------------------------------------------------------
%	BIBLIOGRAPHY
%----------------------------------------------------------------------------------------

%\bibliographystyle{apalike}

%\bibliography{biblio.bib}



\end{document}
